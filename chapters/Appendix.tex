\thispagestyle{plain}
\section{Source Code}
\subsubsection{stereo\_calibrate.cpp}
\begin{verbatim}
#include "opencv2/calib3d/calib3d.hpp"
#include "opencv2/highgui/highgui.hpp"
#include "opencv2/imgproc/imgproc.hpp"
#include "opencv2/core/types.hpp"
#include "opencv2/imgproc.hpp"
#include "opencv2/ximgproc.hpp"

#include <vector>
#include <string>
#include <algorithm>
#include <iostream>
#include <iterator>
#include <stdio.h>
#include <stdlib.h>
#include <ctype.h>
#include <fstream>




using namespace cv;
using namespace std;


//CV Vars
Mat camera1image1;
Mat camera1image2;
Mat camera2image1;
Mat camera2image2;
Size imSize;

Mat distortionCoefficients[2]; //array of matrices for the distortion coefficients, one per camera
Mat rotationMatrix; //a rotation matrix from one camera to the other
Mat translationVector; //a translation vector from one camera to the other
Mat essentialMatrix;
Mat fundamentalMatrix;

Mat cameraMatrices[2];
Mat rotationMatrices[2];
Mat projectionMatrices[2];
Mat disparityToDepth;


//Stereo BM object for creating disparity image
Ptr<StereoBM> bm;

Ptr<StereoSGBM> sgbm;

//Our output disparity maps
Mat disp, disp8, disparityVis;


vector<Point3f> Create3DChessboardCoordinates(Size boardSize, float squareSize);



int main(int argc, char *argv[]) {

	//initialize the size of the board to 6x9
	Size boardSize(6, 9);

	//read in the image files
	string camera1image1fn = "res/left01.jpg";
	string camera1image2fn = "res/left02.jpg";
	string camera2image1fn = "res/right01.jpg";
	string camera2image2fn = "res/right02.jpg";




	const float squareSize = 1.0f;

	//read the images into openCV matrices
	camera1image1 = imread(camera1image1fn, CV_LOAD_IMAGE_GRAYSCALE);
	camera1image2 = imread(camera1image2fn, CV_LOAD_IMAGE_GRAYSCALE);
	camera2image1 = imread(camera2image1fn, CV_LOAD_IMAGE_GRAYSCALE);
	camera2image2 = imread(camera2image2fn, CV_LOAD_IMAGE_GRAYSCALE);

	imSize = camera1image1.size();

	//check that the sizes of the images are the same
	if (camera1image2.size() != imSize || camera1image1.size() != imSize || camera2image2.size() != imSize)
	{
		std::cerr << "All images must be the same size!" << std::endl;
		return -1;
	}


	//we know the images are the same size, now find the chessboard corners in the first image
	//an array of arrays of points for the left camera, one array per image
	vector<vector<Point2f> > camera1ImagePoints(2);
	bool found = findChessboardCorners(camera1image1, boardSize, camera1ImagePoints[0]);//check to see if we can find the chessboard in the first image
	if (!found)
	{
		cerr << "Corners not found in image 1,1" << endl;
		exit(-1);
	}

	found = findChessboardCorners(camera1image2, boardSize, camera1ImagePoints[1]);
	if (!found)
	{
		cerr << "Corners not found in image 1,2" << endl;
		exit(-1);
	}

	//now the same for the right camera
	vector<vector<Point2f> > camera2ImagePoints(2);
	found = findChessboardCorners(camera2image1, boardSize, camera2ImagePoints[0]);//check to see if we can find the chessboard in the first image
	if (!found)
	{
		cerr << "Corners not found in image 2,1" << endl;
		exit(-1);
	}

	found = findChessboardCorners(camera2image2, boardSize, camera2ImagePoints[1]);
	if (!found)
	{
		cerr << "Corners not found in image 2,2" << endl;
		exit(-1);
	}


	//initialize our fake 3D coordinate system, one for each set of images
	vector<vector<Point3f> > objectPoints(2);

	//both are identical, because we're using the same chessboard in each
	objectPoints[0] = Create3DChessboardCoordinates(boardSize, squareSize);
	objectPoints[1] = Create3DChessboardCoordinates(boardSize, squareSize);
	//init the initial camera matrices
	cameraMatrices[0] = Mat::eye(3, 3, CV_64F); //3x3 identity matrix of 64bit floating point numbers(doubles)
	cameraMatrices[1] = Mat::eye(3, 3, CV_64F);


	//init the output matrices for the stereoCalibrate step


	double error = stereoCalibrate(objectPoints, camera1ImagePoints, camera2ImagePoints,
			cameraMatrices[0], distortionCoefficients[0],
			cameraMatrices[1], distortionCoefficients[1],
			imSize, rotationMatrix, translationVector,
			essentialMatrix, fundamentalMatrix,
			CV_CALIB_FIX_ASPECT_RATIO +
			CV_CALIB_ZERO_TANGENT_DIST +
			CV_CALIB_SAME_FOCAL_LENGTH +
			CV_CALIB_RATIONAL_MODEL +
			CV_CALIB_FIX_K3 +
			CV_CALIB_FIX_K4 +
			CV_CALIB_FIX_K5, TermCriteria(CV_TERMCRIT_ITER + CV_TERMCRIT_EPS, 100, 1e-5)); //Termination Criteria (Why move this between 2.4 and 3.0?)




	//now we have the right parameters to rectify the images
	stereoRectify(cameraMatrices[0], distortionCoefficients[0], cameraMatrices[1], distortionCoefficients[1],
			imSize, rotationMatrix, translationVector, rotationMatrices[0], rotationMatrices[1], projectionMatrices[0], projectionMatrices[1],
			disparityToDepth, 0, 0, cvSize(0, 0));


	ofstream outfile;
	outfile.open("res/CalibrationInfo.txt");
	outfile<< imSize.height << ' ' << imSize.width << endl << endl;
	//camera matrices
	for(int i = 0; i < 3; i ++){
		for(int j = 0; j < 3; j ++){
			outfile<< cameraMatrices[0].at<double>(i,j) << ' ';
		}
		outfile << endl;
	}
	outfile << endl;
	for(int i = 0; i < 3; i ++){
		for(int j = 0; j < 3; j ++){
			outfile<< cameraMatrices[1].at<double>(i,j) << ' ';
		}
		outfile << endl;
	}
	outfile << endl;
	//rotation matrices
	for(int i = 0; i < 3; i ++){
		for(int j = 0; j < 3; j ++){
			outfile<< rotationMatrices[0].at<double>(i,j) << ' ';
		}
		outfile << endl;
	}
	outfile << endl;

	for(int i = 0; i < 3; i ++){
		for(int j = 0; j < 3; j ++){
			outfile<< rotationMatrices[1].at<double>(i,j) << ' ';
		}
		outfile << endl;
	}
	outfile << endl;

	//projection matrices
	for(int i = 0; i < 3; i ++){
		for(int j = 0; j < 4; j ++){
			outfile<< projectionMatrices[0].at<double>(i,j) << ' ';
		}
		outfile << endl;
	}
	outfile << endl;

	for(int i = 0; i < 3; i ++){
		for(int j = 0; j < 4; j ++){
			outfile<< projectionMatrices[1].at<double>(i,j) << ' ';
		}
		outfile << endl;
	}
	outfile << endl;

	//distortion coeffiicients
	for(int i = 0; i < 8; i ++){

		outfile<< distortionCoefficients[0].at<double>(i) << endl;
	}
	outfile << endl;

	for(int i = 0; i < 8; i ++){

		outfile<< distortionCoefficients[1].at<double>(i) << endl;
	}
	outfile << endl;

	outfile.close();

	return 0;

}


vector<Point3f> Create3DChessboardCoordinates(Size boardSize, float squareSize) {

	//create a vector of coordinates
	std::vector<Point3f> corners;

	for (int i = 0; i < boardSize.height; i++)
	{
		for (int j = 0; j < boardSize.width; j++)
		{
			//push the individual coordinates of the chessboard corners
			//its a 3D point, so we assume that all points lay on the 0 z plane
			corners.push_back(Point3f(float(j*squareSize),float(i*squareSize), 0));
		}
	}

	return corners;
}
\end{verbatim}

\subsubsection{CVRectify.cpp}
\begin{verbatim}
#include "opencv2/calib3d/calib3d.hpp"
#include "opencv2/highgui/highgui.hpp"
#include "opencv2/imgproc/imgproc.hpp"
#include "opencv2/core/types.hpp"
#include "opencv2/imgproc.hpp"
#include "opencv2/ximgproc.hpp"

#include <iostream>
#include <fstream>


using namespace cv;
using namespace std;

Mat camera1image;
Mat camera2image;
Size imSize;
Mat cameraMatrices[2];
Mat rotationMatrices[2];
Mat projectionMatrices[2];
Mat distortionCoefficients[2];

Ptr<StereoBM> bm;
Ptr<StereoSGBM> sgbm;


void Init_SBM();
void Init_SGBM();

int main(){
	//open file and read in values
	std::ifstream infile;
	infile.open("res/CalibrationInfo.txt");
	int w, h;
	//image size
	infile >> h >> w ;
	imSize.height = h;
	imSize.width = w;

	double a, b, c;
	double arr[3][3];
	//camera matrices
	for(int i = 0; i < 3; i ++){
		infile >> a >> b >> c;
		arr[i][0] = a;
		arr[i][1] = b;
		arr[i][2] = c;
	}
	cameraMatrices[0] = Mat(3, 3, CV_64F, arr);

	double arr1[3][3];
	for(int i = 0; i < 3; i ++){
		infile >> a >> b >> c;
		arr1[i][0] = a;
		arr1[i][1] = b;
		arr1[i][2] = c;
	}
	cameraMatrices[1] = Mat(3, 3, CV_64F, arr1);

	//rotation matrices
	double arr2[3][3];
	for(int i = 0; i < 3; i ++){
		infile >> a >> b >> c;
		arr2[i][0] = a;
		arr2[i][1] = b;
		arr2[i][2] = c;
	}
	rotationMatrices[0] = Mat(3, 3, CV_64F, arr2);

	double arr3[3][3];
	for(int i = 0; i < 3; i ++){
		infile >> a >> b >> c;
		arr3[i][0] = a;
		arr3[i][1] = b;
		arr3[i][2] = c;
	}
	rotationMatrices[1] = Mat(3, 3, CV_64F, arr3);
	//projection matrices
	double d;
	double arr4[3][4];
	for(int i = 0; i < 3; i ++){
		infile >> a >> b >> c >> d;
		arr4[i][0] = a;
		arr4[i][1] = b;
		arr4[i][2] = c;
		arr4[i][3] = d;
	}
	projectionMatrices[0] = Mat(3, 4, CV_64F, arr4);

	double arr5[3][4];
	for(int i = 0; i < 3; i ++){
		infile >> a >> b >> c >> d;
		arr5[i][0] = a;
		arr5[i][1] = b;
		arr5[i][2] = c;
		arr5[i][3] = d;
	}
	projectionMatrices[1] = Mat(3, 4, CV_64F, arr5);

	//distortion coefficients
	double arr6[8];
	for(int i = 0; i < 8; i ++)
	{
		infile >> arr6[i];
	}

	distortionCoefficients[0] = Mat(8, 1, CV_64F, arr6);

	double arr7[8];
	for(int i = 0; i < 8; i ++)
	{
		infile >> arr7[i];
	}

	distortionCoefficients[1] = Mat(8, 1, CV_64F, arr7);

	cout << imSize << endl;
	cout << cameraMatrices[0] << endl;
	cout << cameraMatrices[1] << endl;
	cout << rotationMatrices[0] << endl;
	cout << rotationMatrices[1] << endl;
	cout << projectionMatrices[0] << endl;
	cout << projectionMatrices[1] << endl;
	cout << distortionCoefficients[0] << endl;
	cout << distortionCoefficients[1] << endl;
	infile.close();

//Mats used for remapping images to their rectified selves
	Mat map11, map12, map21, map22;
	initUndistortRectifyMap(cameraMatrices[0], distortionCoefficients[0], rotationMatrices[0], projectionMatrices[0], imSize, CV_16SC2, map11, map12);
	initUndistortRectifyMap(cameraMatrices[1], distortionCoefficients[1], rotationMatrices[1], projectionMatrices[1], imSize, CV_16SC2, map21, map22);

	//init the Stereo Block Matcher, needs only be done once
	Init_SBM();

	Mat img1rectified, img2rectified, disp, dispVis;
	namedWindow("Disparity Map", 1);

	//while(true){
			
		camera1image = imread("res/left02.jpg", CV_LOAD_IMAGE_GRAYSCALE);
		camera2image = imread("res/right02.jpg", CV_LOAD_IMAGE_GRAYSCALE);
		
		remap(camera1image, img1rectified, map11, map12, INTER_LINEAR);
		remap(camera2image, img2rectified, map21, map22, INTER_LINEAR);

		//Calc the Disparity map using Stereo BlockMatching
		bm->compute(img1rectified, img2rectified, disp);
		cv::ximgproc::getDisparityVis(disp, dispVis, 1.0);
		imshow("Disparity Map", dispVis);
		waitKey(0);
	//}
}

void Init_SBM(){
	bm = StereoBM::create(64, 11); //create the StereoBM Object

	//bm->setROI1(); //usable area in rectified image
	//bm->setROI2(roi2);
	bm->setPreFilterCap(31);
	//bm->setBlockSize(9); //block size to check
	//	bm->setMinDisparity(-32);
	bm->setNumDisparities(128); //number of disparities
	bm->setTextureThreshold(32);
	//	bm->setUniquenessRatio(15);
	bm->setSpeckleWindowSize(96);
	bm->setSpeckleRange(64);
	//	bm->setDisp12MaxDiff(1);

}


void Init_SGBM(){
	sgbm = StereoSGBM::create(0, 16, 3); //create the StereoBM Object


	sgbm->setPreFilterCap(63);
	sgbm->setBlockSize(3);
	sgbm->setP1(72);
	sgbm->setP2(256);
	sgbm->setMinDisparity(0);
	sgbm->setNumDisparities(192);
	sgbm->setUniquenessRatio(10);
	sgbm->setSpeckleRange(8);
	sgbm->setDisp12MaxDiff(1);


}
\end{verbatim}
\subsubsection{GLRectify.cpp}
\begin{verbatim}
#include <iostream>
#include <fstream>
#include <vector>

#include <IL/il.h>
#include <IL/ilu.h>
#include <IL/ilut.h>

#include "Common/esUtil.h"
#include "glm/glm.hpp"

#define DEBUG 0

using namespace std;
using namespace glm;

unsigned int imHeight, imWidth;
//Needed GL Vars
mat3 camera1;
mat3 camera2;

mat3 rotation1;
mat3 rotation2;

mat3x4 projection1;
mat3x4 projection2;

//IL stuff
ILuint ImgId;

typedef struct
{
	// Handle to a program object
	GLuint rectifyProgramObject;
	GLuint disparityProgramObject;


	// Attribute locations
	GLint  positionLoc;
	GLint  colorLoc;

	//vertex data
	GLfloat *vertices;
	GLubyte * colors;
	GLuint * indices;
	int numIndices;

} UserData;


void Draw ( ESContext *esContext );
int Init ( ESContext *esContext );
GLfloat * init_VertexInfo();
GLubyte * init_VertexColors(char * filename);
GLuint * genIndices(int picWidth, int picHeight, int * numIndices);
GLuint CreateSimpleTexture2D( );
void ShutDown ( ESContext *esContext );

int main(int argc, char ** argv){


	//open file and read in values
	std::ifstream infile;
	infile.open("res/CalibrationInfo.txt");

	infile >> imHeight >> imWidth;


	double a, b, c;
	//camera matrices
	for(int i = 0; i < 3; i ++){
		infile >> a >> b >> c;
		camera1[i][0] = a;
		camera1[i][1] = b;
		camera1[i][2] = c;
	}

	for(int i = 0; i < 3; i ++){
		infile >> a >> b >> c;
		camera2[i][0] = a;
		camera2[i][1] = b;
		camera2[i][2] = c;
	}

	//rotation matrices
	for(int i = 0; i < 3; i ++){
		infile >> a >> b >> c;
		rotation1[i][0] = a;
		rotation1[i][1] = b;
		rotation1[i][2] = c;
	}

	for(int i = 0; i < 3; i ++){
		infile >> a >> b >> c;
		rotation2[i][0] = a;
		rotation2[i][1] = b;
		rotation2[i][2] = c;
	}

	//projection matrices
	double d;
	for(int i = 0; i < 3; i ++){
		infile >> a >> b >> c >> d;
		projection1[i][0] = a;
		projection1[i][1] = b;
		projection1[i][2] = c;
		projection1[i][3] = d;
	}

	for(int i = 0; i < 3; i ++){
		infile >> a >> b >> c >> d;
		projection2[i][0] = a;
		projection2[i][1] = b;
		projection2[i][2] = c;
		projection2[i][3] = d;
	}

	infile.close();
#if DEBUG
	cout << glm::to_string(camera1) << endl;
	cout << glm::to_string(camera2) << endl;
	cout << glm::to_string(rotation1) << endl;
	cout << glm::to_string(rotation2) << endl;
	cout << glm::to_string(projection1) << endl;
	cout << glm::to_string(projection2) << endl;
#endif


	//done reading in needed vars, now lets init our OpenGl Stuff
	ESContext esContext;
	UserData  userData;

	esInitContext ( &esContext );
	esContext.userData = &userData;

	esCreateWindow ( &esContext, "Simple Texture 2D", imWidth, imHeight, ES_WINDOW_RGB );

	if ( !Init ( &esContext ) )
		return 0;

	esRegisterDrawFunc ( &esContext, Draw );

	esMainLoop ( &esContext );

	ShutDown ( &esContext );
}

void Draw ( ESContext *esContext )
{
	UserData *userData =(UserData *)(esContext->userData);

	// Set the viewport
	glViewport ( 0, 0, esContext->width, esContext->height );

	// Clear the color buffer
	glClear ( GL_COLOR_BUFFER_BIT );

	// Use the program object
	glUseProgram ( userData->rectifyProgramObject );

	// Load the vertex position
	glVertexAttribPointer ( userData->positionLoc, 4, GL_FLOAT,
			GL_FALSE, 1, userData -> vertices);
	// Load the texture coordinate
	glVertexAttribPointer ( userData->colorLoc, 3, GL_BYTE,
			GL_FALSE, 1, userData -> colors);

	glEnableVertexAttribArray ( userData->positionLoc );
	glEnableVertexAttribArray ( userData->colorLoc );


	glDrawElements ( GL_TRIANGLES, userData->numIndices, GL_UNSIGNED_INT, userData->indices );
	eglSwapBuffers(esContext->eglDisplay, esContext->eglSurface);
}

int Init ( ESContext *esContext )
{
	//init the img data
	ilInit();
	ilGenImages(1, &ImgId);


	esContext->userData = malloc(sizeof(UserData));
	UserData *userData = (UserData *)(esContext->userData);

	GLbyte vShaderStr[] =
			"uniform mat4 transformMatrix; \n"
			"uniform mat4 projectionMatrix;\n"
			"uniform mat4 cameraMatrix;    \n"
			"attribute vec4 a_Position;    \n"
			"attribute vec3 a_Color;    \n"
			"varying vec3 v_Color;      \n"
			"void main()                   \n"
			"{                             \n"
			"   gl_Position = a_Position;  \n"
			"   v_Color = a_Color;   \n"
			"}                             \n";

	GLbyte fShaderStr[] =
			"precision mediump float;                            \n"
			"varying vec3 v_Color;                               \n"
			"void main()                                         \n"
			"{                                                   \n"
			"    gl_FragColor = vec4(1.0,1.0,1.0,1.0);           \n"
			"}                                                   \n";

	// Load the shaders and get a linked program object
	userData->rectifyProgramObject = esLoadProgram ((char *)vShaderStr, (char *)fShaderStr );

	// Get the attribute locations
	userData->positionLoc = glGetAttribLocation ( userData->rectifyProgramObject, "a_Position" );
	userData->colorLoc = glGetAttribLocation ( userData->rectifyProgramObject, "a_Color" );

	glClearColor ( 0.39f, 0.58f, 0.92f, 1.0f );

	userData -> vertices =init_VertexInfo();
	userData -> indices =genIndices(imWidth, imHeight, &(userData->numIndices));
	userData -> colors = init_VertexColors("res/left01.jpg");
	return GL_TRUE;
}


GLfloat * init_VertexInfo(){

	GLfloat * tmpVertexInfo = (GLfloat *)malloc(imWidth * imHeight * 4 * sizeof(GLfloat));//init memory for positions


	//indices = genIndices(imWidth, imHeight);

	for(int i = 0; i < imHeight; i ++){
		for(int j = 0; j < imWidth * 4; j+=4){
			int loc = (i*imWidth*4) + j;
			tmpVertexInfo[loc] = ((GLfloat)j/4) / imWidth;
			tmpVertexInfo[loc + 1] = ((GLfloat)i )/ imHeight;
			tmpVertexInfo[loc + 2] = 1.0f;
			tmpVertexInfo[loc + 3] = 1.0f;
		}

	}


	return tmpVertexInfo;
}


GLubyte * init_VertexColors(char * filename){

	ilBindImage(ImgId);
	ilLoadImage(filename);
	ILubyte * tmp = ilGetData();

	/*for(int i = 0; 640 * 480 * 3; i +=3){
		cout << '[' << (int)tmp[i] << ' ' << (int)tmp[i+1] << ' ' << (int)tmp[i+2] << ']' << endl;
	}*/
	cout << "image Loaded" << endl;
	return (GLubyte *)tmp;
}

GLuint * genIndices(int picWidth, int picHeight, int * numIndices){
	vector<int> temp;

	for(int i = 0; i < picHeight; i ++){
		for(int j = 0; j < picWidth -1; j++){
			if(i %2 == 0){//even rows
				temp.push_back(i * picWidth + j);
				temp.push_back((i * picWidth) + j + picWidth);
				temp.push_back(i * picWidth + j + 1);
			}
			else{ //odd rows
				temp.push_back(i * picWidth + j);
				temp.push_back(i * picWidth + j + 1);
				temp.push_back((i-1) * picWidth + j);
			}
		}

	}
	*numIndices = temp.size();

	GLuint * indices = (GLuint *)malloc(temp.size() * sizeof(GLint));
	for(int i = 0; i < temp.size(); i ++){
		indices[i] = temp[i];
	}
	return indices;
}

void ShutDown ( ESContext *esContext )
{
	UserData *userData =(UserData *)(esContext->userData);


	// Delete program object
	glDeleteProgram ( userData->rectifyProgramObject );

	free(esContext->userData);
}
\end{verbatim}
\subsubsection{rectify.vs}
\begin{verbatim}
#version 120
//"Vertex" shader to rectify the left and right image.
//takes as input the transformation matrix generated by stereoRectify()

uniform mat4 transformMatrix;
uniform mat4 projectionMatrix;

attribute vec2 vertexPosition;

varying vec2 texCoord;


main(){
	//put our 2 coordinates into a vec4 and then multiply by our rectification matrix.
	gl_position = vec4(vertexPosition, 1.0f, 1.0f) * transformMatrix * projectionMatrix;
}
\end{verbatim}
\subsubsection{rectify.fs}
\begin{verbatim}
#version 120
//Simple Texturing shader for rectifying


varying vec2 texCoord;

uniform sampler2D Tex1;

main(){

	gl_fragColor = texture(Tex1, texCoord);
}
\end{verbatim}
\subsubsection{disparity.vs}

\subsubsection{disparity.fs}
