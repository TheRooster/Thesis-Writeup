According to the Center for Disease Control (CDC) over 20 million americans suffer vision loss.[1]
While other sensory deficiencies such as hearing loss have had the benefits of computerized assistive
devices, the full extent of assistive devices for the vision impaired include a stick or a trained dog.
We have reached a point, however, where computers have become small enough and powerful enough
that the possibility of creating an assistive device for the vision impaired exists.
If such a device were to exist, it would require the application of real time or near real time
stereo vision. Stereo vision is the process by which a computer takes images from two cameras and
recombines them to extract depth information. Such a process is very computationally expensive,
and the mobility requirements demand that the device performing this computation be of a small
size and low power consumption.
It is not known, however, whether a computer of such small size and low power consumption,
such as the recently released Raspberry Pi 2, can be made to perform stereo vision in a fast enough
manner so as to be useful for these kinds of applications. The purpose of this research will be to
determine whether or not a Raspberry Pi 2 can output a 30 frame per second (fps) or greater depth
map. The process will involve implementing the stereo vision portion of the Open Computer Vision
(OpenCV) library on the Raspberry Pi 2, converting the highly parallel portions of that library to
run on the devices Graphics Processing Unit (GPU), and multi-threading any other portions of the
library where possible.